\documentclass[twoside]{amsart}

\oddsidemargin-0.80cm
\evensidemargin-0.10cm
\topmargin-1.6cm     %I recommend adding these three lines to increase the 
\textwidth17.55cm   %amount of usable space on the page (and save trees)
\textheight23.85cm  

\usepackage[brazilian]{babel}
\usepackage{amsmath}
\usepackage{amssymb}
\usepackage{bbm}
\usepackage{graphics}
\usepackage{mathtools}
\usepackage{hyperref}
\usepackage{physics}

\numberwithin{equation}{section}

\newtheorem{teo}{Teorema}[section]
\newtheorem{defi}{Definição}[section]
\newtheorem{lem}{Lema}[section]

\pagestyle{plain}


\title{
Notas de Teoria Quântica de Campos
}
\author{
  Vicente V. Figueira
       }
\date{\today}

\begin{document}

\maketitle

\tableofcontents

\newpage

\section{Introdução}

\newpage

\section{Transformações de Lorentz}

A grande motivação por trás de iniciar-se uma busca por uma Teoria Quântica de Campos é ao se perceber que 
a Mecânica Quântica de Schrödinger não é compatível com a Relatividade Restrita. Em busca de conciliar uma 
teoria que englobe ambas Mecânica Quântica e Relatividade Restrita somos levados naturalmente a definir 
uma teoria na qual as \emph{funções de onda} são na verdade operadores em um espaço de Hilbert. Naturalmente 
como queremos englobar a relatividade, temos que primeiramente entender os detalhes das transformações de Lorentz.

\subsection{Transformações de Lorentz}

Uma transformação de Lorentz é uma transformação linear das coordenadas que preserva o produto interno relativístico, 
isto é, 

\begin{align*}
    {x'}^{\mu}=\Lambda^{\mu}_{\ \rho}x^\rho+a^\mu
\end{align*}

A condição de preservar o produto interno relativístico pode ser escrita como,

\begin{align*}
    g_{\mu\nu}\dd{{x'}^{\mu}}\dd{{x'}^{\nu}}&=g_{\rho\sigma}\dd{x^{\rho}}\dd{x^\sigma}\\
    g_{\mu\nu}\pdv{{x'}^\mu}{x^\rho}\pdv{{x'}^\nu}{x^\sigma}&=g_{\rho\sigma}\\
    g_{\mu\nu}\Lambda^{\mu}_{\ \rho}\Lambda^{\nu}_{\ \sigma}&=g_{\rho\sigma}
\end{align*}

Transformações desse tipo claramente formam um grupo, pois,

\begin{align*}
    {x''}^\mu&={\bar\Lambda}^\mu_{\ \rho}{x'}^\rho+{\bar a}^\mu\\
    &={\bar\Lambda}^\mu_{\ \rho}\qty(\Lambda^\rho_{\ \nu}x^\nu+ a^\rho)+{\bar a}^\mu\\
    &={\bar\Lambda}^\mu_{\rho}\Lambda^\rho_{\ \nu}x^\nu+\qty({\bar\Lambda}^\mu_{\rho}a^\rho+{\bar a}^\mu)
\end{align*}

Com a lei de transformações então satisfazendo,

\begin{align*}
    T\qty(\bar\Lambda,\bar a)T\qty(\Lambda, a)&=T\qty(\bar\Lambda\Lambda,\bar\Lambda a+\bar a)
\end{align*}

Essas transformações atuam nos próprios pontos do espaço-tempo, porém, nos interessa mais como essas transformações 
afetam os vetores, e operadores, definidos no espaço de Hilbert. Como essas transformações são continuamente deformáveis 
à identidade $\Lambda=\mathbbm 1$ e $a=0$, segue que podemos encontrar uma representação para a atuação dessas transformações no espaço de Hilbert como um operador unitário 

\end{document}