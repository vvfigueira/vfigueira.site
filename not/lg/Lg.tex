\documentclass[twoside]{amsart}

\usepackage[brazilian]{babel}
\usepackage{csquotes}
\usepackage{amsmath}
\usepackage{amssymb}
\usepackage{bbm}
\usepackage{graphics}
\usepackage{mathtools}
\usepackage[hidelinks]{hyperref}
\usepackage{physics}
\usepackage{enumitem}
\usepackage{slashed}
\usepackage{tensor}
\usepackage[lmargin=0.5cm,rmargin=0.5cm, tmargin =1cm,bmargin =1cm]{geometry}

\AtBeginDocument{\renewcommand*{\hbar}{{\mkern-1mu\mathchar'26\mkern-8mu\textnormal{h}}}}
\AtBeginDocument{\newcommand{\e}{\textnormal{e}}}
\AtBeginDocument{\newcommand{\im}{\textnormal{i}}}
\AtBeginDocument{\newcommand{\luz}{\textnormal{c}}}
\AtBeginDocument{\newcommand{\grav}{\textnormal{G}}}
\AtBeginDocument{\newcommand{\kb}{{\textnormal{k}_{\textnormal{B}}}}}
\newcommand{\Dd}[1]{\mathcal D #1}
\newcommand{\Det}[1]{\textup{Det} #1}
\newcommand{\cqd}{\hfill$\blacksquare$}

\numberwithin{equation}{section}

\newtheorem{teo}{Teorema}[section]
\newtheorem{defi}{Definição}[section]
\newtheorem{lem}{Lema}[section]
\newtheorem{hip}{Hipótese}[subsection]

\pagestyle{plain}

\AddToHook{cmd/section/before}{\clearpage}

\title{
Srednicki Capítulo 3
}
\author{
  Vicente V. Figueira
       }
\date{\today}

\begin{document}

\maketitle

\tableofcontents

%%%%%%%%%%%%%%%%%%%%%%%%%%%%%%%%%%%%%%%%%%%%%%%%%%%%%%%%%%%%%

\section{}

Transformações de coordenadas gerais são,

\begin{align*}
    x^\mu\rightarrow \tensor{\Lambda}{^\mu_\nu}x^\nu+a^\mu
\end{align*}

Note que se fazemos duas transformações,

\begin{align*}
    x^\mu&\rightarrow \tensor{{\Lambda'}}{^\mu_\nu}\qty(\tensor{\Lambda}{^\nu_\rho}x^\rho+a^\nu)+{a'}^\mu\\
    &\rightarrow \tensor{{\Lambda'}}{^\mu_\nu}\tensor{\Lambda}{^\nu_\rho}x^\rho+\tensor{{\Lambda'}}{^\mu_\nu}a^\nu+{a'}^\mu
\end{align*}

Isso implica que as transformações de coordenadas formam um grupo, satisfazendo a regra de composição,

\begin{align*}
    \qty(\Lambda',a')\qty(\Lambda,a)&=\qty(\Lambda'\Lambda,\Lambda'a+a')
\end{align*}

Transformações infinitesimais são,

\begin{align*}
    \tensor{\Lambda}{^\mu_\nu}&=\tensor{\delta}{^\mu_\nu}+\tensor{\omega}{^\mu_\nu}+\mathcal O\qty(\omega^2)
\end{align*}

Mas,

\begin{align*}
    \Lambda^\textnormal{T}\Lambda&=\mathbbm 1\\
    \qty(\tensor{\delta}{^\mu_\nu}+\tensor{\omega}{^\mu_\nu})^\textnormal{T}\qty(\tensor{\delta}{^\nu_\rho}+\tensor{\omega}{^\nu_\rho})&=\tensor{\delta}{^\mu_\rho}\\
    \omega_{\mu\nu}&=-\omega_{\nu\mu}
\end{align*}

\end{document}