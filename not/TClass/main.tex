\documentclass[twoside]{amsart}

\usepackage[brazilian]{babel}
\usepackage{csquotes}
%\usepackage[sorting=none, style=verbose-inote, backend=biber]{biblatex}
\usepackage{amsmath}
\usepackage{amssymb}
\usepackage{bbm}
\usepackage{graphics}
\usepackage{mathtools}
\usepackage[hidelinks]{hyperref}
\usepackage{physics}
\usepackage{enumitem}
\usepackage{slashed}
\usepackage{tensor}
\usepackage[lmargin=0.5cm,rmargin=0.5cm, tmargin =1cm,bmargin =1cm]{geometry}

\AtBeginDocument{\renewcommand*{\hbar}{{\mkern-1mu\mathchar'26\mkern-8mu\textnormal{h}}}}
\AtBeginDocument{\newcommand{\e}{\textnormal{e}}}
\AtBeginDocument{\newcommand{\im}{\textnormal{i}}}
\AtBeginDocument{\newcommand{\luz}{\textnormal{c}}}
\AtBeginDocument{\newcommand{\grav}{\textnormal{G}}}
\AtBeginDocument{\newcommand{\kb}{{\textnormal{k}_{\textnormal{B}}}}}
\newcommand{\Dd}[1]{\mathcal D #1}
\newcommand{\Det}[1]{\textup{Det} #1}
\newcommand{\sgn}[1]{\mbox{sgn}\qty(#1)}
\newcommand{\cqd}{\hfill$\blacksquare$}

\numberwithin{equation}{section}

\newtheorem{teo}{Teorema}[section]
\newtheorem{defi}{Definição}[section]
\newtheorem{lem}{Lema}[section]
\newtheorem{hip}{Hipótese}[subsection]

\pagestyle{plain}

\AddToHook{cmd/section/before}{\clearpage}

%\addbibresource{ref.bib}

\title{
Notas de Mecânica Clássica
}
\author{
  Vicente V. Figueira
       }
\date{\today}

\begin{document}

\maketitle

\tableofcontents

%%%%%%%%%%%%%%%%%%%%%%%%%%%%%%%%%%%%%%%%%%%%%%%%%%%%%%%%%%%%%

%\begin{refsection}
\section{Introdução}

\subsection{Fundamentos da Teoria Clássica}

Podemos pensar na Física como um conjunto de várias teorias, cada qual com seu limite de 
validade, e, estas historicamente foram nomeadas segundo estes, ou, segundo o fenômeno 
característico que se propõem a modelar. A Teoria Clássica, ou Mecânica Clássica como 
alguns chamam, está entre as primeiras teorias física a ser proposta, e como seu nome 
sugere, está intencionada a descrever fenômenos \emph{não extremos}, seja do ponto de 
vista de energia, número de constituintes ou dimensões do sistema. Questões como qual 
os valores numéricos dos reais limites da teoria só podem serem sanadas com comparação 
experimental dos valores teóricos.

A formulação da Teoria Clássica tem início com observações experimentais e experimentos 
mentais, que são \emph{confiáveis} para extrair informações \emph{médias} em sistemas de 
energia, número de constituintes e dimensões \emph{médias}. O primeiro ápice da formulação 
Clássica foi devido a Newton, seguido por Lagrange e Hamilton. Porém, estamos aqui interessados 
em desenvolver uma abordagem via postulados para sedimentar quais informações estão sendo 
assumidas em nossa teoria.

Ao tentar dar início sobre quais são as hipóteses ocultas feitas sobre uma teoria, devemos 
pensar primeiramente qual são os principais agentes dinâmicos, que por experiência devem ser 
posições e velocidades, estes que são vetores, logo, deve haver um espaço vetorial abaixo, 
e via experiência, este é um espaço de três dimensões. Certamente, novamente por experiência,  
também há uma quantidade que flui naturalmente, chamada comumente de \emph{tempo}. 
Naturalmente, eu, em meu referencial, utilizando de aparatos adequados posso medir intervalos 
espaciais e temporais à vontade, porém, outra pessoa, distante de mim, poderia por sua vez 
também estar interessada em realizar as mesmas medições que eu estou fazendo, se ela realiza 
as medições do seu referencial, e eu realizo as medições do meu referencial, há alguma 
correlação entre estas? Bem, naturalmente se para referenciais distintos não houvesse nenhum 
tipo de correlação entre os valores medidos toda a física cairia por terra, visto que não 
haveria nenhum modo de comparar medidas e testar teorias físicas, deve portanto haver ao menos 
alguma classe de referenciais tais que as coordenadas espaciais e temporais possuam correlação 
entre os dois observadores. Observadores pertencentes a esta classe especial de referenciais que concordam com a física descrita são chamados de observadores \emph{inerciais}, devemos portanto descrever qual propriedades queremos que um referencial inercial possua. O requisito mais intuitivo é de que observadores inerciais percebem objetos que não interagem movem-se com velocidade constante, a princípio mudanças de coordenadas do tipo de translação são triviais de serem tratadas, apenas se fazendo, $${\bar x}^\mu=x^\mu+a^\mu$$
Nosso interesse esta em mudanças de referenciais não dadas por translações, como boost e rotações, essas atuam de forma linear como, $${\bar x}^\mu={\textup L}^\mu_{\ \nu}x^\nu$$
No caso de uma rotação pura ao redor de um eixo $\boldsymbol\theta$ com módulo $\theta$, assumimos que a componente temporal não é alterada, assim, o vetor mais genérico linear capaz de ser formado com $\vb x,\boldsymbol\theta$ e $\theta$ de modo a preservar a norma $\vb{\bar x}=\vb x$ é,

\begin{align}
    \vb{\bar x}&=\frac{\boldsymbol\theta\cdot\vb x}{\theta^2}\boldsymbol\theta+\qty(\vb x-\frac{\boldsymbol\theta\cdot\vb x}{\theta^2}\boldsymbol\theta)\cos\theta-\frac{\boldsymbol\theta\times \vb x}{\theta}\sin\theta
\end{align}

Logo a matriz $\textup L$ é $${\textup L}^{0}_{\ 0}=1,\ {\textup L}^0_{\ i}={\textup L}^{i}_{\ 0}=0,\ {\textup L}^{i}_{\ j}={\textup R}^{i}_{\ j}$$
Com,
\begin{align}
    {\textup R}^{i}_{\ j}&=\frac{\theta_i\theta_{j}}{\theta^2}+\qty(\delta_{ij}-\frac{\theta_{i}\theta_j}{\theta^2})\cos\theta+\frac{\epsilon_{ijk}\theta_k}{\theta}\sin\theta
\end{align}

Para um boost, temos que construir escalares e vetores com $\vb v$ e $\vb x$, a construção linear mais geral é,

\begin{align}
    {\bar t}&=a\qty(v)t+b\qty(v)\vb v\cdot \vb x\\
    \vb {\bar x}&=c\qty(v)\vb x+d\qty(v)\frac{\vb v\cdot \vb x}{v^2}\vb v+e\qty(v)t\vb v
\end{align}

A condição $\vb x=\vb v t$ deve implicar $\vb{\bar x}=0$, logo a restrição é,

\begin{align}
    c\qty(v)+d\qty(v)+e\qty(v)&=0
\end{align}

A transformação inversa deve valer para $\vb {\bar v}=-\vb v$, o que implica em

\begin{align}
    c^2&=1\\
    a^2-ebv^2&=1\\
    e^2-ebv^2&=1\\
    ea+e^2&=0\\
    ba+be&=0
\end{align}

Tomamos $c=1$ pois $c=-1$ corresponde a uma rotação. Como $e\neq 0$ segue que $a=-e$, logo,

\begin{align}
    b&=\frac{1-a^2}{av^2}\\
    c&=1\\
    d&=a-1\\
    e&=-a
\end{align}

Como dois boost seguidos deve corresponder a um único boost, chegamos em

\begin{align}
    \frac{va\qty(v)}{\bar wa\qty(\bar w)}\qty(1-a^2\qty(\bar w))&=\frac{\bar wa\qty(\bar w)}{va\qty(v)}\qty(1-a^2\qty(v))\\
    \frac{1-a^2\qty(v)}{v^2a^2\qty(v)}&=\frac{1-a^2\qty(\bar w)}{{\bar w}^2a^2\qty({\bar w})}=K\\
    a\qty(v)&=\frac{\pm 1}{\sqrt{1+Kv^2}}=\pm \gamma\qty(v),\ \ b\qty(v)=\frac{\pm K}{\sqrt{1+Kv^2}}=\pm K\gamma\qty(v) 
\end{align}

Logo a transformação é,

\begin{align}
    {\bar t}&=\pm\gamma\qty(v)\qty(t+K\vb v \cdot \vb x)\\
    \vb {\bar x}&= \vb x+\frac{\pm\gamma\qty(v)-1}{v^2}\qty(\vb v \cdot \vb x)\vb v\mp\gamma\qty(v)t \vb v
\end{align}

Resta apenas fixar o sinal de $a\qty(v)$ e o valor de $K$, o caso $K=0$ faz com que $\gamma=1$ e retorna as relações da relatividade de Galileu. Para casos $K\neq 0$, apenas é importante o sinal de $K$ e não seu valor absoluto, visto que podemos sempre mudar as unidades de tempo para obter,

\begin{align}
    {\bar t}&=\pm\gamma\qty(v)\qty(t+\sgn{K}\norm{K}\vb v \cdot \vb x)\\
    \vb {\bar x}&= \vb x+\frac{\pm\gamma\qty(v)-1}{v^2}\qty(\vb v \cdot \vb x)\vb v\mp\gamma\qty(v)t \vb v\\
    \norm{K}^{-\frac12}{\bar t}&=\pm\gamma\qty(v)\qty(\norm{K}^{-\frac12}t+\sgn{K}\norm{K}^\frac12\vb v \cdot \vb x)\\
    \vb {\bar x}&= \vb x+\frac{\pm\gamma\qty(v)-1}{\norm{K}v^2}\qty(\norm{K}^\frac12\vb v \cdot \vb x)\norm{K}^\frac12\vb v\mp\gamma\qty(v)\norm{K}^{-\frac12}t \norm{K}^\frac12\vb v 
\end{align}

Redefinindo $t\norm{K}^{-\frac12}\rightarrow t$,

\begin{align}
    {\bar t}&=\pm\gamma\qty(v)\qty(t+\sgn{K}\vb v \cdot \vb x)\\
    \vb {\bar x}&= \vb x+\frac{\pm\gamma\qty(v)-1}{v^2}\qty(\vb v \cdot \vb x)\vb v\mp\gamma\qty(v)t \vb v
\end{align}

Para ter uma intuição sobre o que o sinal de $K$ significa vamos fazer dois boost seguidos,

\begin{align}
    {\bar{\bar t}}&=\pm\gamma\qty(w)\qty({\bar t}+\sgn{K}\vb w\cdot\vb{\bar x})\\
    &=\pm\gamma\qty(w)\qty(\pm\gamma\qty(v)\qty{t+\sgn{K}\vb v\cdot\vb x}+\sgn{K}\vb w\cdot\qty[\vb x+\frac{\pm\gamma\qty(v)-1}{v^2}\qty(\vb v \cdot \vb x)\vb v\mp\gamma\qty(v)t \vb v])\\
    &=\gamma\qty(w)\gamma\qty(v)\qty(t+\sgn{K}\vb v\cdot\vb x\pm\frac{\sgn K}{\gamma\qty(v)}\vb w\cdot \vb x+\sgn K\frac{\gamma\qty(v)\mp1}{v^2\gamma\qty(v)}\qty(\vb w\cdot \vb v)\vb v\cdot \vb x-\sgn K t\vb w \cdot \vb v)\\
    &=\gamma\qty(w)\gamma\qty(v)\qty(1-\sgn K\vb w\cdot \vb v)\qty(t+\frac{\sgn K}{v^2\gamma\qty(v)\qty(1-\sgn K\vb w\cdot \vb v)} \vb x\cdot \qty{v^2\gamma\qty(v)\vb v\pm v^2\vb w+\qty(\gamma\qty(v)\mp 1)\qty(\vb w\cdot \vb v)\vb v})
\end{align}

Fazendo as identificações,

\begin{align}
    \pm\gamma\qty(u)&=\gamma\qty(w)\gamma\qty(v)\qty(1-\sgn K\vb w\cdot \vb v)\\
    \vb u &=\frac{v^2\gamma\qty(v)+\qty(\gamma\qty(v)\mp1)\vb w\cdot\vb v}{v^2\gamma\qty(v)\qty(1-\sgn K\vb w\cdot\vb v)}\vb v\pm\frac{1}{\gamma\qty(v)\qty(1-\sgn K \vb w\cdot \vb v)}\vb w
\end{align}

Para $a\qty(v)=\pm\gamma\qty(v)$ ser real com $\sgn K=-1$ somos obrigados a ter $\norm{v}< 1$ para todos os referenciais inerciais, assim $1-\sgn K \vb w\cdot \vb v>0$, que por sua vez nos exige tomar $a\qty(v)=+\gamma\qty(v)$. Analogamente $\sgn K=0$ também fixa $a\qty(v)=+\gamma\qty(v)$, outra opção restante é $\sgn K=+1$ que não restringe os valores de $\norm{v}$, assim $a\qty(v)$ pode assumir valores tanto positivos quanto negativos, valor de $a\qty(v)$ negativo implica em reversão temporal, como estamos assumindo uma transformação passiva não é esperado que resulte em uma inversão na direção temporal, logo, fixamos $K\leq 0$ como casos \emph{físicos}. A Teoria Clássica está interessada no caso $K=0$, para o qual as transformações são da forma,

\begin{align}
    \gamma&=1\\
    {\bar t}&=t\\
    \vb {\bar x}&=\vb x- t\vb v\\
    \vb u&=\vb v+\vb w
\end{align}

Portanto temos que o tempo é na verdade uma quantidade absoluta, gostaríamos de introduzir uma métrica para qual as normas seja invariantes perante a rotações e boosts, isto é,

\begin{align*}
    \qty(Bx)^{\textup T}gBx=x^{\textup T}B^\textup{T}gBx
\end{align*}

Isto é, a métrica deve satisfazer, $B^{\textup T}gB=g$, se supomos que,

\begin{align*}
    x=\mqty(t&\vb x^{\textup T})^{\textup T}
\end{align*}    

Logo,

\begin{align*}
    \mqty(a&{\vb b}^{\textup T}\\ \vb c&\mathbbm M)&=\mqty(1&-\vb v^{\textup T}\\ \vb 0&\mathbbm 1)\mqty(a&{\vb b}^{\textup T}\\ \vb c&\mathbbm M)\mqty(1&\vb 0^\textup T\\-\vb v&\mathbbm 1)\\
    \mqty(a&{\vb b}^{\textup T}\\ \vb c&\mathbbm M)&=\mqty(a-\vb v^{\textup T}\vb c&\vb b^{\textup T}-\vb v^{\textup T}\mathbbm M\\ \vb c&\mathbbm M)\mqty(1&\vb 0^\textup T\\-\vb v&\mathbbm 1)\\
    \mqty(a&{\vb b}^{\textup T}\\ \vb c&\mathbbm M)&=\mqty(a-\vb v^{\textup T}\vb c-\vb b^{\textup T}\vb v+\vb v^{\textup T}\mathbbm M\vb v&\vb b^{\textup T}-\vb v^{\textup T}\mathbbm M\\ \vb c-\mathbbm M \vb v&\mathbbm M)\\
\end{align*}

Donde concluímos que,

\begin{align}
    \mathbbm M =\vb  0,\ \vb b=-\vb c
\end{align}

Esta métrica deve possuír uma inversa, para isso,

\begin{align}
    \textup{Det } g&=-b_1\textup {Det }\mqty(-b_1&0&0\\-b_2&0&0\\-b_3&0&0)+b_2\textup {Det }\mqty(-b_1&0&0\\-b_2&0&0\\-b_3&0&0)-b_3\textup {Det }\mqty(-b_1&0&0\\-b_2&0&0\\-b_3&0&0)=0
\end{align}

Logo esta não é inversível, não é possível portanto estabelecer uma métrica que seja invariante pelos boosts. Uma maneira de contornar este problema é introduzir outra coordenada $s$, assim um vetor é,

\begin{align}
    x=\mqty(\vb x^{\textup T}&t&s)^{\textup T}
\end{align}

A transformação por boost deve ser claramente,

\begin{align}
    \mqty(\vb x-\vb v t\\t\\ s')&=\mqty(\mathbbm 1&-\vb v&\vb 0\\ \vb 0^{\textup T }&1&0\\\vb v^{\textup T}a\qty(v)&b\qty(v)&c\qty(v))\mqty(\vb x\\ t\\ s)
\end{align}

A transformação inversa deve ser aquela com $-\vb v$, então,

\begin{align}
    \mqty(\mathbbm 1&\vb 0&\vb 0\\\vb 0^{\textup T}&1&0\\\vb 0^{\textup T}&0&1)&=\mqty(\mathbbm 1&\vb v&\vb 0\\ \vb 0^{\textup T }&1&0\\-\vb v^{\textup T}a\qty(v)&b\qty(v)&c\qty(v))\mqty(\mathbbm 1&-\vb v&\vb 0\\ \vb 0^{\textup T }&1&0\\\vb v^{\textup T}a\qty(v)&b\qty(v)&c\qty(v))\\
    &=\mqty(\mathbbm 1&\vb 0&\vb 0\\ \vb 0^{\textup T }&1&0\\-\vb v^{\textup T}a\qty(v)+\vb v^{\textup T}a\qty(v)c\qty(v)&v^2a\qty(v)+b\qty(v)+b\qty(v)c\qty(v)&c^2\qty(v))
\end{align}

Isso fixa $c=1$ e $b=-v^2\frac a2$, o boost é dado por,

\begin{align}
    \mqty(\mathbbm 1&-\vb v&\vb 0\\ \vb 0^{\textup T }&1&0\\\vb v^{\textup T}a&-v^2\frac a2&1)
\end{align}

A métrica deve ser da forma,

\begin{align}
    g&=\mqty(\mathbbm M&\vb b&\vb c\\\vb b^{\textup T}&d&e\\\vb c^{\textup T}&e&f)
\end{align}

Requerendo que preserve os boosts,

\begin{align}
    &\mqty(\mathbbm M&\vb b&\vb c\\\vb b^{\textup T}&d&e\\\vb c^{\textup T}&e&f)\\
    &=\mqty(\mathbbm 1&\vb 0&\vb v a\\ -\vb v^{\textup T }&1&-v^2\frac a2\\\vb 0^{\textup T}&0&1)\mqty(\mathbbm M&\vb b&\vb c\\\vb b^{\textup T}&d&e\\\vb c^{\textup T}&e&f)\mqty(\mathbbm 1&-\vb v&\vb 0\\ \vb 0^{\textup T }&1&0\\\vb v^{\textup T}a&-v^2\frac a2&1)\\
    &=\mqty(\mathbbm 1&\vb 0&\vb v a\\ -\vb v^{\textup T }&1&-v^2\frac a2\\\vb 0^{\textup T}&0&1)\mqty(\mathbbm M+a\vb c\vb v^{\textup T}&-\mathbbm M\vb v+\vb b-v^2\frac a2\vb c&\vb c\\ \vb b^{\textup T}+ae\vb v^{\textup T}&-\vb b^{\textup T}\vb v+d-v^2\frac a2 e&e\\\vb c^{\textup T}+af\vb v^{\textup T}&-\vb c^{\textup T}\vb v+e-v^2\frac a2 f&f)\\
    &=\mqty(\mathbbm M+a\vb c\vb v^{\textup T}+a\vb v\vb c^{\textup T}+a^2f\vb v\vb v^{\textup T}&-\mathbbm M \vb v+\vb b-\frac12 av^2\vb c-a\vb v\vb c^{\textup T}\vb v+ae\vb v-\frac12 a^2fv^2\vb v&\vb c+af\vb v\\ -\vb v^{\textup T}\mathbbm M-a\vb v^{\textup T}\vb c\vb v^{\textup T}+\vb b^{\textup T}+ae\vb v^{\textup T}-\frac12 av^2\vb c^{\textup T}-\frac12 a^2fv^2\vb v^{\textup T}&\vb v^{\textup T}\mathbbm M\vb v-\vb v^{\textup T}\vb b+\frac12 av^2\vb v^{\textup T}\vb c-\vb b^{\textup T}\vb v+d-\frac12av^2e+\frac12av^2\vb c^{\textup T}\vb v-\frac12 av^2e+\frac14 a^2v^4f&-\vb v^{\textup T}\vb c+e-\frac12 av^2f\\ \vb c^{\textup T}+af\vb v^{\textup T}&-\vb c^{\textup T}\vb v+e-\frac12 av^2f&f)
\end{align}

Que implica em,

\begin{align}
    a\vb c\vb v^{\textup T}+a\vb v\vb c^{\textup T}+a^2f\vb v\vb v^{\textup T}&=0
\end{align}

Logo, $f=0$, e isso implica em $\vb c=0$. Assim,

\begin{align}
    &\mqty(\mathbbm M&\vb b&\vb 0\\\vb b^{\textup T}&d&e\\\vb 0^{\textup T}&e&0)\\
    &=\mqty(\mathbbm M&-\mathbbm M \vb v+\vb b+ae\vb v&\vb 0\\ -\vb v^{\textup T}\mathbbm M+\vb b^{\textup T}+ae\vb v^{\textup T}&\vb v^{\textup T}\mathbbm M\vb v-\vb v^{\textup T}\vb b-\vb b^{\textup T}\vb v+d-av^2e&e\\ \vb 0^{\textup T}&e&0)
\end{align}

Que por sua vez implica em,

\begin{align}
    \mathbbm M \vb v&= ae\vb v,\ \Rightarrow \mathbbm M=ae\mathbbm 1,\ \Rightarrow \vb b=\vb 0
\end{align}

Então a métrica fica como,

\begin{align}
    g=\mqty(ae\mathbbm 1&\vb 0&\vb 0\\\vb 0^{\textup T}&d&e\\\vb 0^{\textup T}&e&0 )=\mqty(ae &0&0&0&0\\0&ae&0&0&0\\0&0&ae&0&0\\0&0&0&d&e\\0&0&0&e&0)
\end{align}

Esta deve possuir uma inversa, logo, seu determinante não pode ser nulo,

\begin{align}
    \textup{Det }g&=ae\textup{Det }\mqty(ae&0&0&0\\0&ae&0&0\\0&0&d&e\\0&0&e&0)\\
    &=\qty(ae)^2\textup{Det }\mqty(ae&0&0\\0&d&e\\0&e&0)\\
    &=\qty(ae)^3\textup{Det }\mqty(d&e\\e&0)=-e^5a^3
\end{align}

Gostaríamos que $\norm{\textup{Det }g}=1$, assim, fixamos $\norm{e^5a^3}=1$, a escolha de $a$ é apenas uma redefinição de unidades, logo escolhemos $a=-1$ pois será mais conveniente futuramente, portanto, resta $\norm{e}=1$, escolhemos $e=-1$ que preserva o sinal positivo da parte espacial na métrica. Resta apenas o parâmetro $d$ indeterminado, que renomearemos para $K$

\begin{align}
    g=\mqty(\mathbbm 1&\vb 0&\vb 0\\\vb 0^{\textup T}&K&-1\\\vb 0^{\textup T}&-1&0 )
\end{align}

A norma de um vetor $A=\mqty(\vb A&A^4&A^5)$ é,

\begin{align}
    \vb A^2+K{A^4}^2-2A^4A^5
\end{align}

O vetor posição é certamente,

\begin{align}
    x=\mqty(\vb x&t&s),\ \ x\cdot x=\vb x^2+Kt^2-2ts
\end{align}

E podemos então definir um vetor velocidade, 

\begin{align}
    u=\dv{x}{t}=\mqty(\vb u&1&\dv{s}{t})
\end{align}

Note que em um referencial no qual acompanhamos um corpo que se move com velocidade instantânea $\vb v$,

\begin{align}
    x'&=\mqty(\vb x-\vb vt&t&s-\vb v^{\textup T}\vb x+\frac12 v^2t)\\
    \dd{x}'&=\mqty(\dd{\vb x}-\dv{\vb x^{\textup T}}{t}\dd{t}&\dd{t}&\dd{s}-\dv{\vb x^{\textup T}}{t}\dd{\vb x}+\frac12 \dv{\vb x^{\textup T}}{t}\dv{\vb x}{t}\dd{t})\\
    \dd{x}'&=\mqty(\vb 0&\dd{t}&\dd{s}-\frac12 \dv{\vb x^{\textup T}}{t}\dv{\vb x}{t}\dd{t})
\end{align}

Mas como no referencial do próprio corpo nenhuma quantidade pode depender de sua velocidade, somos levado a concluir que,

\begin{align}
    \dd{s}=\frac12\vb u^2\dd{t}
\end{align}

Ou seja, em um referencial que observa o corpo se mover,

\begin{align}
    x=\mqty(\vb x&t&\frac12\int\limits_{0}^{t}\dd{t'}\vb u^2\qty(t'))
\end{align}

Dessa forma,

\begin{align}
    u&=\mqty(\vb u&1 &\frac12 \vb u^2)\\
    mu=p&=\mqty(m\vb u&m&\frac m2 \vb u^2)\\
    p&=\mqty(\vb p&m&E)
\end{align}

Um corpo livre deve obedecer a lei,

\begin{align}
    \dv{p_\mu}{t}&=0
\end{align}

Qualquer alteração desta chamamos de força, o que nos leva a escrever,

\begin{align}
    \dv{p_\mu}{t}&=F_\mu\\
    \dv{}{t}\qty[m\dv{x_\mu}{t}]&=F_\mu\\
    \dv{}{t}\qty[m\dv{x_\mu}{t}]&=-\partial_\mu\Phi+\dv{}{t}\qty[\pdv{\Phi}{\dv{x^\mu}{t}}]    
\end{align}

Que pode ser gerado variacionalmente por,

\begin{align}
    S&=\int\dd{t}\qty{\frac m2\dv{x^\mu}{t}\dv{x_\mu}{t}-\Phi\qty(x, \dv{x}{t})}
\end{align}

O fato de que $p_\mu F^\mu=0$, implica em,

\begin{align}
    S&=\int\dd{t}\qty{\frac m2\dv{x^\mu}{t}\dv{x_\mu}{t}-\dv{x_\mu}{t}A^\mu\qty(x)}
\end{align}

Porém temos que,

\begin{align}
    \dv{x_\mu}{t}\dv{x^\mu}{t}&=K
\end{align}

Para incorporar este vínculo precisamos adicionar um multiplicador de Lagrange fazendo,

\begin{align*}
    S&=\int\dd{t}\qty{\frac\alpha 2\dv{x_\mu}{t}\dv{x^\mu}{t}-\dv{x_\mu}{t}A^\mu\qty(x)+\frac{m^2K}{2\alpha}}
\end{align*}

Claro que $\alpha=\alpha\qty(t)$, assim as equações de movimento se tornam,

\begin{align*}
    \frac12\dv{x_\mu}{t}\dv{x^\mu}{t}-\frac{m^2K}{2\alpha^2}&=0\\
    \dv{}{t}\qty{\frac\alpha 2\dv{x^\mu}{t}-A^\mu}&=-\dv{x_\nu}{t}\partial^\mu A^{\nu}\\
    \dv{}{t}\qty{\sqrt{\frac{K}{\dv{x_\mu}{t}\dv{x^\mu}{t}}}\frac m 2\dv{x^\mu}{t}-A^\mu}&=-\dv{x_\nu}{t}\partial^\mu A^{\nu}\\
\end{align*}

Claro que isso só faz sentido se exigirmos invariância de reparametrização,

\begin{align*}
    S&=\int\dd{t}\qty{\frac\alpha 2\dv{x_\mu}{t}\dv{x^\mu}{t}-\dv{x_\mu}{t}A^\mu\qty(x)+\frac{m^2K}{2\alpha}}\\
    &=\int\dd{\lambda}\dv{t}{\lambda}\qty{\frac\alpha 2\dv{\lambda}{t}\dv{\lambda}{t}\dv{x_\mu}{\lambda}\dv{x^\mu}{\lambda}-\dv{\lambda}{t}\dv{x_\mu}{\lambda}A^\mu\qty(x)+\frac{m^2K}{2\alpha}}\\
    &=\int\dd{\lambda}\qty{\frac\alpha 2\dv{\lambda}{t}\dv{x_\mu}{\lambda}\dv{x^\mu}{\lambda}-\dv{x_\mu}{\lambda}A^\mu\qty(x)+\frac{m^2K}{2\alpha\dv{\lambda}{t}}}\\
\end{align*}

Donde retiramos a lei de transformação via reparametrização,

\begin{align}
    \alpha\qty(\lambda)&=\alpha\qty(t)\dv{\lambda}{t}
\end{align}

Assim podemos escrever de uma maneira covariante e invariante por reparametrização, tomando $K=1$ por simplicidade,

\begin{align}
    S&=\int\dd{\lambda}\qty{\frac\alpha 2\dot{x}_\mu\dot{x}^\mu-\dot{x}_\mu A^\mu+\frac{m^2}{2\alpha}}
\end{align}

Note que a equação de movimento escrita de forma geral é,

\begin{align}
    \dv{}{\lambda}\qty[\frac{m\dot {x}^\mu}{\sqrt{\dot x_\nu\dot x^\nu}}]&=\dv{}{\lambda}A^\mu-\partial^\mu\qty(\dot x_\nu A^\nu)
\end{align}

Definimos como momento $p^\mu=\frac{m\dot {x}^\mu}{\sqrt{\dot x_\nu\dot x^\nu}}=\alpha \dot x^\mu$

Vamos escrever a equação de movimento geral para um objeto supondo que estamos em um referencial tal que $A^\mu\mqty(\vb A\qty(\vb x)&A^4\qty(\vb x)&A^5\qty(\vb x))$,

\begin{align}
    \dv{\vb p}{t}&=\dot x_\nu\qty[\partial^\nu \vb A-\grad A^\nu]\\
    &=\qty(\dot {\vb x}\cdot\grad)\vb A-\grad\qty(\dot{\vb x}\cdot\vb A)+\dot x_4\grad A^5+\dot x_5\grad A^4-K\dot x_4\grad A^4\\
    &=\dot{\vb x}\times\qty(\grad\times \vb A)+\dot x_4\grad A^5+\dot x_5\grad A^4-K\dot x_4\grad A^4\\
    \dv{p^4}{t}=\dv{m}{t}&=\dot x_\nu\qty[\partial^\nu A^4-\partial^4A^\nu]\\
    &=\dot x_\nu\partial^\nu A^4\\
    &=\qty(\dot {\vb x}\cdot\grad )A^4\\
    \dv{p^5}{t}=\dv{E}{t}&=\dot x_\nu\qty[\partial^\nu  A^5-\partial^5 A^\nu]\\
    &=\dot x_\nu\partial^\nu  A^5\\
    &=\qty(\dot {\vb x}\cdot \grad)  A^5
\end{align}

Note que como $A$ deve ser vetor por Galileu,

\begin{align}
    {A'}^4\qty(\vb x ')&=A^4\qty(\vb x)\\
    A^4\qty(\vb x-\vb vt)&=A^4\qty(\vb x)\rightarrow A^4\qty(\vb x)=\textup{cte}
\end{align}

E,

\begin{align}
    {A'}^5\qty(\vb x')&=A^5\qty(\vb x)\\
    {A}^5\qty(\vb x-\vb vt)+\vb A\qty(\vb x-\vb vt)\cdot \vb v&=A^5\qty(\vb x)
\end{align}

E também,

\begin{align}
    {\vb A}'\qty(\vb x')&=\vb A\qty(\vb x)\\
    \vb A\qty(\vb x-\vb v t)&=\vb A\qty(\vb x)
\end{align}

%\printbibliography[heading=subbibliography]
%\end{refsection}

%%%%%%%%%%%%%%%%%%%%%%%%%%%%%%%%%%%%%%%%%%%%%%%%%%%%%%%%%%%%%

%\begin{refsection}
\section{Formalismo Lagrangiano}



%\printbibliography[heading=subbibliography]
%\end{refsection}

%%%%%%%%%%%%%%%%%%%%%%%%%%%%%%%%%%%%%%%%%%%%%%%%%%%%%%%%%%%%%

%\begin{refsection}
\section{Formalismo Hamiltoniano}


%\printbibliography[heading=subbibliography]
%\end{refsection}

%%%%%%%%%%%%%%%%%%%%%%%%%%%%%%%%%%%%%%%%%%%%%%%%%%%%%%%%%%%%%

%\begin{refsection}
\section{Formalismo de Hamilton-Jacobi}

%\printbibliography[heading=subbibliography]
%\end{refsection}

%%%%%%%%%%%%%%%%%%%%%%%%%%%%%%%%%%%%%%%%%%%%%%%%%%%%%%%%%%%%%

%\begin{refsection}
\section{Teoria Clássica de Campos}



%\printbibliography[heading=subbibliography]
%\end{refsection}

%%%%%%%%%%%%%%%%%%%%%%%%%%%%%%%%%%%%%%%%%%%%%%%%%%%%%%%%%%%%%o  

\end{document}