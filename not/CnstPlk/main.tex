\documentclass[twoside]{amsart}

\usepackage[brazilian]{babel}
\usepackage{csquotes}
\usepackage[sorting=none, style=verbose-inote, backend=biber]{biblatex}
\usepackage{amsmath}
\usepackage{amssymb}
\usepackage{bbm}
\usepackage{graphics}
\usepackage{mathtools}
\usepackage[hidelinks]{hyperref}
\usepackage{physics}
\usepackage{enumitem}
\usepackage{slashed}
\usepackage{tensor}
\usepackage[lmargin=0.5cm,rmargin=0.5cm, tmargin =1cm,bmargin =1cm]{geometry}

\AtBeginDocument{\renewcommand*{\hbar}{{\mkern-1mu\mathchar'26\mkern-8mu\textnormal{h}}}}
\AtBeginDocument{\newcommand{\e}{\textnormal{e}}}
\AtBeginDocument{\newcommand{\im}{\textnormal{i}}}
\AtBeginDocument{\newcommand{\luz}{\textnormal{c}}}
\AtBeginDocument{\newcommand{\grav}{\textnormal{G}}}
\AtBeginDocument{\newcommand{\kb}{{\textnormal{k}_{\textnormal{B}}}}}
\newcommand{\Dd}[1]{\mathcal D #1\: }
\newcommand{\Det}[1]{\textup{Det} #1}
\newcommand{\sgn}[1]{\mbox{sgn}\qty(#1)}
\newcommand{\cqd}{\hfill$\blacksquare$}

\numberwithin{equation}{section}

\newtheorem{teo}{Teorema}[section]
\newtheorem{defi}{Definição}[section]
\newtheorem{lem}{Lema}[section]
\newtheorem{hip}{Hipótese}[subsection]

\pagestyle{plain}

\AddToHook{cmd/section/before}{\clearpage}

%\addbibresource{ref.bib}

\title{
Origem da Constante de Planck
}
\author{
  Vicente V. Figueira
       }
\date{\today}

\begin{document}

\maketitle

\tableofcontents

%%%%%%%%%%%%%%%%%%%%%%%%%%%%%%%%%%%%%%%%%%%%%%%%%%%%%%%%%%%%%

%\begin{refsection}
\section{Introdução}

Introdução

%\printbibliography[heading=subbibliography]
%\end{refsection}

%%%%%%%%%%%%%%%%%%%%%%%%%%%%%%%%%%%%%%%%%%%%%%%%%%%%%%%%%%%%%

%\begin{refsection}
\section{Cálculo}
  
Da teoria da informação sabemos que precisamos maximizar a quantidade 
$$\mathcal I\qty[p]=-\int\Dd {x}p\qty[x]\ln\qty(\frac{p\qty[x]}{m\qty[x]})$$ Com as devidas condições 
de contorno. No nosso caso, a condição de contorno é da \textbf{ação média ser a ação clássica}, isto é,
$$\int\Dd xp\qty[x]S\qty[x]=S[x_{\textnormal{cl}}]$$ Com é claro $x_{\textnormal{cl}}\qty(t)$ sendo 
definido por, $$\fdv{S}{x\qty(t)}\qty[x_{\textnormal{cl}}]=0$$ Logo, o que devemos maximizar é,

\begin{align*}
    \mathcal I\qty[p]&=-\int\Dd x\ln\qty(\frac{p\qty[x]}{m\qty[x]})-\alpha\qty(\int\Dd xp\qty[x]S\qty[x]-S\qty[x_{\textnormal{cl}}])\\
    &\begin{cases}
        \fdv{\mathcal I}{p\qty[x]}&=0=-\ln\qty(\frac{p\qty[x]}{m\qty[x]})-1-\alpha S\qty[x]\\
        \pdv{\mathcal I}{\alpha}&=0=\int\Dd xp\qty[x]S\qty[x]-S[x_{\textnormal{cl}}]
    \end{cases}\\
    p\qty[x]&=m\qty[x]\exp\qty(-\alpha S\qty[x]-1)\\
    S\qty[x_{\textnormal{cl}}]&=\int\Dd{x}m\qty[x]\exp\qty(-\alpha S\qty[x]-1)S\qty[x]
\end{align*}
  
Definimos assim $$m\qty[x]=\qty(\int\Dd x\exp\qty(-\alpha S\qty[x]-1))^{-1}$$ Dessa forma, podemos escrever,
$$Z\qty(\alpha)=\int\Dd x\exp\qty(-\alpha S\qty[x])$$ Onde a condição se torna
$$-\pdv{}{\alpha}\ln\qty[Z\qty(\alpha)]=S\qty[x_\textnormal{cl}]$$

Resume-se o cálculo à calcular $Z\qty(\alpha)$. Impomos os vínculos do caminho como sendo, $$x\qty(t_{\textnormal i})=x_{\textnormal i},\ x\qty(t_{\textnormal f})=x_{\textnormal f}$$
Dividimos em $N$ partes com $x_0=x_{\textnormal i},\ x_N=x_{\textnormal f},\ \Delta t=\frac{t_{\textnormal f}-t_{\textnormal i}}{N}$,

\begin{align*}
    Z\qty(\alpha)&=\lim\limits_{N\rightarrow \infty}\int\prod\limits_{j=1}^{N-1}\dd{x_j}\exp\qty{-\frac{\alpha m}{2}\sum\limits_{k=1}^N\frac{\qty(x_k-x_{k-1})^2}{{\Delta t}^2}\Delta t}\\
    &=\lim\limits_{N\rightarrow \infty}\qty(\frac{2\Delta t}{\alpha m})^{\frac{N-1}{2}}\int\prod\limits_{j=1}^{N-1}\dd{y_j}\exp\qty{-\sum\limits_{k=1}^N\qty(y_{k}-y_{k-1})^2}\\
    &=\lim\limits_{N\rightarrow \infty}\qty(\frac{2\Delta t}{\alpha m})^{\frac{N-1}{2}}\frac{\pi^{\frac{N-1}{2}}}{N^{\frac12}}\exp\qty{-\frac{\qty(y_N-y_0)^2}{N}}\\
    &=\lim\limits_{N\rightarrow \infty}\qty(\frac{2\Delta t\pi}{\alpha m})^{\frac{N-1}{2}}\frac{1}{\sqrt{N}}\exp\qty{-\frac{\alpha m}{2}\frac{\qty(x_{\textnormal f}-x_{\textnormal i})^2}{t_{\textnormal f}-t_{\textnormal i}}}\\
    &=\exp\qty{-\alpha S\qty[x_{\textnormal{cl}}]}\lim\limits_{N\rightarrow \infty}\qty(\frac{2\pi\qty(t_{\textnormal f}-t_{\textnormal i})}{\alpha m})^{\frac{N-1}{2}}N^{-\frac N2}
\end{align*}

O que é claramente divergente, pois é necessário uma normalização da integral funcional, a normalização tem 
origem no fato de que,

\begin{align*}
    \int\limits_{{\mathbb R}^D}\dd[D]{\vb x}\exp\qty(- \frac\pi a\vb x\cdot\vb x)&=a^{\frac D2}
\end{align*}

Para o limite $D\rightarrow\infty$,

\begin{align*}
    a^\infty&=\begin{cases}
        0&:\quad0<a<1\\
        1&:\quad a=1\\
        \infty&:\quad a>1
    \end{cases}
\end{align*}

Que não é contínua no parâmetro $a$. Uma das possibilidades é tomar a medida,

\begin{align*}
    \int\Dd{x}\exp\qty(-\frac\pi ax^2)=\lim\limits_{D\rightarrow \infty}\int\limits_{{\mathbb R}^D}\prod\limits_{j=1}^D\qty(a^{-\frac12}\dd{x_j})\exp\qty(- \frac\pi a\vb x\cdot\vb x)=1   
\end{align*}

Logo para cada $\dd{x_j}$ é necessário adicionar um fator de normalização $R_N$ dependente da quantidade 
de intervalos de divisão. Como $R_N$ deve ter dimensão de inverso de comprimento, a dependência em $\alpha, m, t_{\textnormal f}-t_{\textnormal i}$ é fixada, 
a dependência adicional em $N$ é tal que previne a divergência da integral, a definir temos uma função arbitrária 
adimensional da variável $$\lambda = \frac{\alpha m\qty(x_{\textnormal f}-x_{\textnormal i})^2}{2\qty(t_{\textnormal f}-t_{\textnormal i})}=\alpha S\qty[x_{\textnormal{cl}}]$$

\begin{align*}
    Z\qty(\alpha)&=\exp\qty{-\alpha S\qty[x_{\textnormal{cl}}]}\lim\limits_{N\rightarrow \infty}\qty(\frac{2\pi\qty(t_{\textnormal f}-t_{\textnormal i})}{\alpha m})^{\frac{N-1}{2}}N^{-\frac N2}R^{N-1}_N\\
    Z\qty(\alpha)&=\exp\qty{-\alpha S\qty[x_{\textnormal{cl}}]}\lim\limits_{N\rightarrow \infty}\qty(\frac{2\pi\qty(t_{\textnormal f}-t_{\textnormal i})}{\alpha m})^{\frac{N-1}{2}}N^{-\frac N2}\qty(\sqrt{\frac{\alpha m}{2\pi\qty(t_{\textnormal f}-t_{\textnormal i})}})^{N-1}\qty(N^{\frac{N}{2\qty(N-1)}})^{N-1}F\qty(\alpha S\qty[x_{\textnormal{cl}}])\\
    Z\qty(\alpha)&=\exp\qty{-\alpha S\qty[x_{\textnormal{cl}}]}F\qty(\alpha S\qty[x_{\textnormal {cl}}])\\
    \ln Z&=-\alpha S\qty[x_{\textnormal{cl}}]+\ln F\\
    -\pdv{}{\alpha}\ln Z&=S\qty[x_{\textnormal{cl}}]-\frac{1}{F}\pdv{}{\alpha}F \buildrel!\over=S\qty[x_{\textnormal{cl}}]\Rightarrow F\equiv 1
\end{align*}

Portanto a densidade de probabilidade é,

$$p\qty[x]=\exp\qty(-\alpha S\qty[x]+\alpha S\qty[x_{\textnormal{cl}}])$$

Outro cálculo possível de ser feito é o da posição média em um tempo $t_{\textnormal i}<t<t_{\textnormal f}$,

\begin{align*}
    \expval{x\qty(t)}&=\int\Dd xx\qty(t)\exp\qty(-\frac{\alpha m}{2}\int\limits_{t_{\textnormal i}}^{t_{\textnormal f}}\dd{t'}\dot x^2\qty(t'))\exp\qty(\alpha S\qty[x_{\textnormal{cl}}])\\
    &=\exp\qty(\alpha S\qty[x_{\textnormal{cl}}])\lim\limits_{N\rightarrow\infty}R_N^{N-1}\int\prod\limits_{j=1}^{N-1}\dd{x_j}x_m\exp\qty(-\frac{\alpha m}{2\Delta t}\sum\limits_{k=1}^N\qty(x_k-x_{k-1})^2)\\
    &=\exp\qty(\alpha S\qty[x_{\textnormal{cl}}])\lim\limits_{N\rightarrow\infty}R_N^{N-1}\qty(\frac{2\Delta t}{\alpha m})^{\frac N2}\int\prod\limits_{j=1}^{N-1}\dd{y_j}y_m\exp\qty(-\sum\limits_{k=1}^N\qty(y_k-y_{k-1})^2)\\
    &=\exp\qty(\alpha S\qty[x_{\textnormal{cl}}])\lim\limits_{N\rightarrow\infty}R_N^{N-1}\qty(\frac{2\Delta t}{\alpha m})^{\frac N2}\frac{\pi^{\frac{m-1}{2}}}{m^{\frac12}}\frac{\pi^{\frac{N-m-1}{2}}}{\qty(N-m)^{\frac12}}\int\dd{y_m}y_m\exp\qty(-\frac{1}{m}\qty(y_m-y_0)^2-\frac{1}{N-m}\qty(y_N-y_m)^2)\\
    &=\exp\qty(\alpha S\qty[x_{\textnormal{cl}}])\lim\limits_{N\rightarrow\infty}R_N^{N-1}\qty(\frac{2\Delta t}{\alpha m})^{\frac N2}\frac{\pi^{\frac{N-2}{2}}}{m^{\frac12}\qty(N-m)^{\frac12}}\times\\
    &\quad\times\int\dd{y_m}y_m\exp\qty(-\frac{N}{m\qty(N-m)}\qty(y_m-\frac{\qty(my_N+\qty(N-m)y_0)}{N})^2+\frac{\qty(my_N+\qty(N-m)y_0)^2}{Nm\qty(N-m)}-\frac{y_0^2}{m}-\frac{y_N^2}{N-m})\\
    &=\exp\qty(\alpha S\qty[x_{\textnormal{cl}}])\lim\limits_{N\rightarrow\infty}R_N^{N-1}\qty(\frac{2\Delta t}{\alpha m})^{\frac N2}\frac{\pi^{\frac{N-2}{2}}}{m^{\frac12}\qty(N-m)^{\frac12}}\sqrt{\frac{\pi m\qty(N-m)}{N}}\frac{\qty(N-m)y_0+my_N}{N}\exp\qty(-\frac1N\qty(y_N-y_0)^2)\\
    &=\lim\limits_{N\rightarrow\infty}\qty(\sqrt{\frac{\alpha m}{2\pi\qty(t_{\textnormal f}-t_{\textnormal i})}})^{N-1}\qty(N^{\frac{N}{2\qty(N-1)}})^{N-1}\qty(\frac{2\qty(t_{\textnormal f}-t_{\textnormal i})}{\alpha mN})^{\frac N2}\frac{\pi^{\frac{N-1}{2}}}{\sqrt N}\sqrt{\frac{\alpha m N}{2\qty(t_{\textnormal f}-t_{\textnormal i})}}\frac{\qty(t_{\textnormal f}-t)x_{\textnormal i}+\qty(t-t_{\textnormal i})x_{\textnormal f}}{t_{\textnormal f}-t_{\textnormal i}}\\
    &=\frac{\qty(t_{\textnormal f}-t)x_{\textnormal i}+\qty(t-t_{\textnormal i})x_{\textnormal f}}{t_{\textnormal f}-t_{\textnormal i}}
\end{align*}

Como esperado para o caminho clássico. Podemos calcular agora a variância da posição,

\begin{align*}
    \expval{x^2\qty(t)}&=\int\Dd xx^2\qty(t)\exp\qty(-\frac{\alpha m}{2}\int\limits_{t_{\textnormal i}}^{t_{\textnormal f}}\dd{t'}\dot x^2\qty(t'))\exp\qty(\alpha S\qty[x_{\textnormal{cl}}])\\
    &=\exp\qty(\alpha S\qty[x_{\textnormal{cl}}])\lim\limits_{N\rightarrow\infty}R_N^{N-1}\qty(\frac{2\Delta t}{\alpha m})^{\frac {N+1}{2}}\frac{\pi^{\frac{N-2}{2}}}{m^{\frac12}\qty(N-m)^{\frac12}}\times\\
    &\quad\times\int\dd{y_m}y^2_m\exp\qty(-\frac{N}{m\qty(N-m)}\qty(y_m-\frac{\qty(my_N+\qty(N-m)y_0)}{N})^2-\frac1N\qty(y_N-y_0)^2)\\
    &=\lim\limits_{N\rightarrow\infty}\qty(\sqrt{\frac{\alpha m}{2\pi\qty(t_{\textnormal f}-t_{\textnormal i})}})^{N-1}N^\frac{N}{2}\qty(\frac{2\qty( t_{\textnormal f}-t_{\textnormal i})}{N\alpha m})^{\frac{N+1}{2}}\frac{\pi^{\frac{N-2}{2}}}{m^{\frac12}\qty(N-m)^{\frac12}}\sqrt{\frac{\pi m\qty(N-m)}{N}}\qty(\frac{m\qty(N-m)}{2N}+\frac{\qty(my_N+\qty(N-m)y_0)^2}{N^2})\\
    &=\lim\limits_{N\rightarrow\infty}\frac{2\qty( t_{\textnormal f}-t_{\textnormal i})}{\alpha m}\frac{1}{N}\qty(N\frac{\qty(t-t_{\textnormal i})\qty(t_{\textnormal f}-t)}{2\qty(t_{\textnormal f}-t_{\textnormal i})^2}+\frac{\alpha m N}{2\qty(t_{\textnormal f}-t_{\textnormal i})}\frac{\qty(\qty(t-t_{\textnormal i})x_{\textnormal f}+\qty(t_{\textnormal f}-t)x_{\textnormal i})^2}{\qty(t_{\textnormal f}-t_{\textnormal i})^2})\\
    &=\expval{x\qty(t)}^2+\frac{\qty(t-t_{\textnormal i})\qty(t_{\textnormal f}-t)}{\alpha m\qty(t_{\textnormal f}-t_{\textnormal i})}
\end{align*}

De modo que a variância é, $$\expval{x^2\qty(t)}-\expval{x\qty(t)}^2=\frac{\qty(t-t_{\textnormal i})\qty(t_{\textnormal f}-t)}{\alpha m\qty(t_{\textnormal f}-t_{\textnormal i})}$$

Para a velocidade,

$$\expval{\dot x\qty(t)}=\dv{}{t}\expval{x\qty(t)}=\frac{x_{\textnormal f}-x_{\textnormal i}}{t_{\textnormal f}-t_{\textnormal i}}$$

%\printbibliography[heading=subbibliography]
%\end{refsection}
  
%%%%%%%%%%%%%%%%%%%%%%%%%%%%%%%%%%%%%%%%%%%%%%%%%%%%%%%%%%%%%
  

\end{document}